\documentclass[landscape,a4paper]{article} %12pt
\usepackage{multicol}
\usepackage[english]{babel}
\usepackage[utf8]{inputenc}
\usepackage{amsmath}
\usepackage{amssymb}
\usepackage{graphicx}
\usepackage[landscape,margin=2.5cm]{geometry}
\usepackage{tabularx}
\usepackage{longtable}
\usepackage{rotating}
\usepackage{listings}
\usepackage[usenames,dvipsnames]{xcolor}
\usepackage{hyperref}

\clubpenalty10000
\widowpenalty10000
\displaywidowpenalty=10000

\newsavebox\ltmcbox

\hypersetup
{
    pdfauthor={Tobias Knopp and Martin M\"oddel},
    pdfsubject={Magnetic Particle Imaging Data Storage Format},
    pdftitle={MDF: Magnetic Particle Imaging Data Format},
    pdfkeywords={MDF, MPI, HDF5},
    pdfproducer={Section for Biomedical Imaging Hamburg University of Technology, Section for Biomedical Imaging University Medical Center Hamburg-Eppendorf},
    pdfcreator={Section for Biomedical Imaging Hamburg University of Technology, Section for Biomedical Imaging University Medical Center Hamburg-Eppendorf}
}

\lstdefinelanguage{MDF}%
  {morekeywords={%
    Bool,Int64,Float64,Number,String%
      },%
   sensitive=true,%
}[keywords,comments,strings]%

\lstset{%
    language         = MDF,
    basicstyle       = \ttfamily,
    numbers=left, 
    columns=fullflexible,
    numberstyle=\small\ttfamily\color{Gray},
    stepnumber=1,              
    numbersep=10pt, 
    numberfirstline=true, 
    numberblanklines=true, 
    tabsize=4,
    lineskip=-1.5pt,
    extendedchars=true,
    breaklines=true,
    keywordstyle     = \bfseries\color{blue},
    stringstyle      = \color{magenta},
    commentstyle     = \color{ForestGreen},
    showstringspaces = false,
    showtabs=false,
    upquote=false,
    texcl=true % interpet comments as LaTeX
}

\newcommand{\inl}[1]{\lstinline[columns=fixed]{#1}}
\newcommand{\inltab}[1]{{\ttfamily\bfseries\color{blue}#1}}
\newcommand{\inlvar}[1]{{\ttfamily#1}}

\begin{document}

\title{MDF: Magnetic Particle Imaging Data Format}
\newcommand{\version}{2.0.0-pre}

\author{
T.~Knopp$^{1,2}$, T.~Viereck$^3$, G.~Bringout$^4$, M.~Ahlborg$^5$, A.~von~Gladiss$^5$, C.~Kaethner$^5$, A.~Neumann$^5$, P.~Vogel$^6$, J.~Rahmer$^7$, M.~M\"oddel$^{1,2}$ \\ \\
$^1$Section for Biomedical Imaging, University Medical Center Hamburg-Eppendorf, Germany\\
$^2$Institute for Biomedical Imaging, Hamburg University of Technology, Germany\\
$^3$Institute of Electrical Measurement and Fundamental Electrical Engineering, TU Braunschweig, Germany\\
$^4$Physikalisch-Technische Bundesanstalt, Berlin, Germany\\
$^5$Institute of Medical Engineering, University of  Lübeck, Germany\\
$^6$Department of Experimental Physics 5 (Biophysics), University of Würzburg, Germany\\
$^7$Philips GmbH Innovative Technologies, Research Laboratories, Hamburg, Germany
}

\maketitle
\begin{center}
Version \textbf{\version}
\end{center}

\begin{abstract}
Magnetic particle imaging (MPI) is a tomographic method to determine the spatio-temporal distribution of magnetic nanoparticles. In this document, a file format for the standardized storage of MPI and magnetic particle spectroscopy (MPS) data is introduced. The aim of the Magnetic Particle Imaging Data Format (MDF) is to provide a coherent way of exchanging MPI and MPS data acquired with different devices worldwide. The focus of the MDF is on sequence parameters, measurement data, calibration data, and reconstruction data. The format is based on the hierarchical document format in version 5 (HDF5). This document discusses the MDF version \version, which is not backward compatible with version 1.x.y.
\end{abstract}

\begin{multicols}{2}		

\section{Introduction} \label{Sec:Introduction}

The purpose of this document is to introduce a file format for exchanging Magnetic Particle Imaging (MPI) and Magnetic Particle Spectroscopy (MPS) data. The Magnetic Particle Imaging Data Format (MDF) is based on the hierarchical document format (HDF) in version 5 \cite{hdf5}. HDF5 is able to store multiple datasets within a single file providing a powerful and flexible data container. To allow an easy exchange of MPI data, one has to specify a naming scheme within HDF5 files which is the purpose of this document. In order to create and access HDF5 data, an Open Source C library is available that provides dynamic access from most programming languages. MATLAB supports HDF5 by its functions \inlvar{h5read} and \inlvar{h5write}. For Python, the \inlvar{h5py} package exists, and the Julia programming language provides access to HDF5 files via the \inlvar{HDF5} package. For languages based on the .NET framework, the \inlvar{HDF5DotNet} library is available.

The MDF is mainly focused on storing measurement data, calibration data, or reconstruction data together with the corresponding sequence parameters and metadata. Even though it is possible to combine measurement data and reconstruction data into a single file, it is recommended to use a single file for each of the following dataset types:
\begin{enumerate}
\setlength{\itemsep}{0pt}
\item Measurement data
\item Calibration data
\item Reconstruction data
\end{enumerate}

\subsection{Datatypes \& Storage Order}

MPI parameters are stored as regular \textit{HDF5 datasets}. \textit{HDF5 attributes} are not used in the current specification of the MDF. For most datasets, a fixed datatype is used, for example the drive-field amplitudes are stored as \inl{H5T_NATIVE_DOUBLE} values. Whenever a data type has a big- and little-endian version, the little-endian data type should be used. For our convenience, we refer to the HDF5 datatypes \inl{H5T_STRING}, \inl{H5T_NATIVE_DOUBLE}, and \inl{H5T_NATIVE_INT64} as \inltab{String}, \inltab{Float64} and \inltab{Int64}. Boolean data is stored as \inl{H5T_NATIVE_INT8}, which we refer to as \inltab{Int8}. 

The datatype of the MPI measurement and calibration data offers more freedom and is denoted by \inltab{Number}, which can be any of the following HDF5 data types: \inl{H5T_NATIVE_FLOAT}, \mbox{\inl{H5T_NATIVE_DOUBLE},} \inl{H5T_NATIVE_INT8}, \inl{H5T_NATIVE_INT16}, \inl{H5T_NATIVE_INT32}, \inl{H5T_NATIVE_INT64} or a complex number as defined below. 

Since storing complex data in HDF5 is not standardized, we define a compound datatype \inl{H5T_COMPOUND} in HDF5 with fields \inl{r} and \inl{i} using one of the above mentioned floating point types to represent the real and the imaginary part of a complex number. This representation was chosen because it is also the default behavior for complex numbers in Python using numpy and h5py. \inltab{Complex128} refers to the complex compound type with base type \inl{H5T_NATIVE_DOUBLE}.

For later identification of a data set, we store three Universally Unique Identifiers (UUIDs) (RFC~4122)~\cite{leach2005universally} in its canonical textual representation as 32 hexadecimal digits, displayed in five groups separated by hyphens \inlvar{8-4-4-4-12} as for example \inlvar{ee94cb6d-febf-47d9-bec9-e3afa59bfaf8}. For the generation of the UUIDs, we recommend to use version 4 of the UUID specification.

Whenever multidimensional data is stored dimensions are arranged in a way that cache is utilized best for fast reconstruction or fast frame selection for example. The leading dimension in the MDF specification is slowest to access and the last dimension is the fastest to access (contiguous memory access). That also means that dependent on the memory layout of your programming language the order of the dimensions will be just as in the MDF specifications (row major) or reversed (col major). Please take this into account when reading or writing MDF files, which includes the usage of HDF5 viewers.

\subsection{Units}

With one exception, physical quantities are given in SI units. The magnetic field strength is reported in T$\mu_0^{-1} = 4 \pi$Am$^{-1}\mu_0^{-1}$. This convention has been proposed in the first MPI publication \cite{Gleich2005} and consistently been used in most MPI related publications. The aims of this convention are to report the numbers on a Tesla scale, which most readers with a background in magnetic resonance imaging are familiar with.

\subsection{Optional Parameters}

The MDF has 8 main groups in the root directory and 2 sub-groups. We distinguish between optional and non-optional groups as well as optional, non-optional, and conditional parameters. 

Any optional parameter can be omitted, whereas any non-optional parameter in a non-optional group is mandatory. Conditional parameters are linked to Boolean parameters and have to be provided, if these parameters are true and can be omitted if they are false. If a parameter is optional, non-optional, or conditional is indicated by yes, no, or the corresponding Boolean parameter.

If a group is optional all of its parameters may be omitted, if this group is not used. The groups \inlvar{/}, \inlvar{/study}, \inlvar{/experiment}, \inlvar{/scanner}, \inlvar{/acquisition} contain mostly metadata and are mandatory. The \inlvar{/tracer} group is only mandatory if magnetic material has been placed in the MPI system. The groups \inlvar{/measurement}, \inlvar{/calibration}, and \inlvar{/reconstruction} are all optional. In case of calibration measurements, the \inlvar{/calibration} group is mandatory. The reconstruction data is stored in \inlvar{/reconstruction}. 

\subsection{Parameter Extension}

Occasionally, it is necessary to store additional specific parameters or metadata that are not covered in the specifications, such as the temperature of the room in which your MPI device is operated. In this case, you are free to add new parameters to any of the existing groups. Moreover, if necessary, you are also free to introduce new groups. In order to be able to distinguish these datasets and groups from the specified ones, we recommend to use the prefix ``\inlvar{\_}'' for all parameters and groups. As an example, one could add a new group \inlvar{/\_room} that includes the dataset \inlvar{\_temperature}.

\subsection{Naming Convention}

Several parameters within the MDF are linked in dimensionality. We use short variable names to indicate these connections. The following table describes the meaning of each variable name used.
\setbox\ltmcbox\vbox{
	\makeatletter\col@number\@ne
	\noindent \begin{longtable}{p{0.12\columnwidth}p{0.8\columnwidth}} 
		\textbf{Variable} & \textbf{Number of}  \\ \hline
		$A$ &  tracer materials/injections for multi-color MPI \\ \hline
		& \\
		$N$ &  acquired frames, same as a spatial position for calibration\\ \hline
		$O$ &  acquired frames w/o background frames\\ \hline
		$J$ &  focus-field patches\\ \hline
		&\\
		$C$ &  receive channels \\ \hline
		$D$ &  drive-field channels \\ \hline
		$F$ &  frequencies describing the drive-field waveform \\ \hline
		$U$ &  sampling points describing a custom drive-field waveform\\ \hline
		&\\
		$V$ &  points sampled at receiver during one patch (product of drive field period, number of periods, and number of averages)\\ \hline
		$W$ &  sampling points containing processed data (\mbox{$W=V$} if no frequency selection or bandwidth reduction has been applied)\\ \hline
		$K$ &  frequencies describing the processed data ($K=V/2+1$ if no frequency selection or bandwidth reduction has been applied)\\ \hline
		&\\
		$Q$ &  frames in the reconstructed MPI data set\\ \hline
		$P$ &  voxels in the reconstructed MPI data set\\ \hline
		$S$ &  channels in the reconstructed MPI data set\\ \hline
	\end{longtable}
	\unskip
	\unpenalty
	\unpenalty}
\unvbox\ltmcbox

\subsection{Contact}

If you find mistakes in this document or the specified file format or if you want to discuss extensions or improvements to this specification, please open an issue on GitHub:\\
\hspace*{1cm}\url{https://github.com/MagneticParticleImaging/MDF}\\
As the file format is versionized it will be possible to extend it for future needs of MPI. The current version discussed in this document is version \version.

\subsection{arXiv}
As of version 1.0.1, the most recent release of these specifications can also be also found at:\\
\hspace*{1cm}\url{https://arxiv.org/abs/1602.06072}\\
If you use MDF, please cite us using the arXiv reference, which is also available for download as \texttt{MDF.bib} from GitHub.


\subsection{Code examples}		 
  		  
If you want to get a basic impression of how to handle MDF files you can visit the gitub repository of the MDF project:\\		
  \hspace*{1cm}\url{https://github.com/MagneticParticleImaging/MDF}\\	
There you will find the directory example, which contains a code example written in Julia, MATLAB, and Python. More details can be found in the \texttt{README} of the repository.
 
\subsection{Reference Implementation}		 

A reference implementation for a high level MDF access is available at:\\
  \hspace*{1cm}\url{https://github.com/MagneticParticleImaging/MPIFiles.jl}\\	
\inlvar{MPIFiles.jl} is a package written in the programming language Julia \cite{Bezanson2012,Bezanson2014,Bezanson2014a}. It can read MDF V1, MDF V2, and the dataformat of Bruker MPI systems using a common interface. It also provides  functions to convert MDF V1 into MDF V2.

\end{multicols}

\clearpage

\section{Data (group: \inlvar{/})}
 \setlength\extrarowheight{5pt}

\begin{multicols}{2}
\paragraph{Remarks:} Within the root group, the metadata about the file itself is stored. Within several subgroups, the metadata about the experimental setting, the MPI tracer, and the MPI scanner can be provided. The actual data is stored in dedicated groups about measurement data and reconstruction data.
\end{multicols}

\noindent \begin{tabularx}{\columnwidth}{lllllX} 
\textbf{Parameter} & \textbf{Type} & \textbf{Dim} & \textbf{Unit/Format} & \textbf{Optional} & \textbf{Description} \\ \hline 
\inlvar{version} & \inltab{String} & 1 & \version & no & Version of the file format \\ \hline
\inlvar{uuid} & \inltab{String} & 1 & 3170fdf8-f8e1-4cbf-ac73-41520b41f6ee & no & Universally Unique Identifier (RFC 4122) of MDF file \\ \hline 
\inlvar{time} & \inltab{String} & 1 & yyyy-mm-ddThh:mm:ss.ms & no & UTC creation time of MDF data set \\ \hline
\end{tabularx}


\subsection{Study Description (group: \inlvar{/study/}, non-optional)}

\begin{multicols}{2}
	\paragraph{Remarks:} A study is supposed to group a series of experiments to support, refute, or validate a hypothesis. The study group should contain \inlvar{name}, \inlvar{number}, \inlvar{uuid}, and \inlvar{description} of the study.
\end{multicols}

\noindent \begin{tabularx}{\columnwidth}{lllllX} 
\textbf{Parameter} & \textbf{Type} & \textbf{Dim} & \textbf{Unit/Format} & \textbf{Optional} & \textbf{Description} \\ \hline 
\inlvar{name} & \inltab{String} & 1 & & no & Name of the study \\ \hline
\inlvar{number} & \inltab{Int64} & 1 & & no & Number of the study\\ \hline
\inlvar{uuid} & \inltab{String} & 1 & 295258fe-b650-4e5f-96db-b83f11089a6c & no & Universally Unique Identifier (RFC 4122) of study \\ \hline 
\inlvar{description} & \inltab{String} & 1 & & no & Short description of the study \\ \hline
\end{tabularx}


\subsection{Experiment Description (group: \inlvar{/experiment/}, non-optional)}

\begin{multicols}{2}
\paragraph{Remarks:} For each experiment within a study a \inlvar{name}, \inlvar{number}, \inlvar{uuid}, and \inlvar{description} have to be provided. Additionally, the name of the \inlvar{subject} imaged and the flag \inlvar{isSimulation} indicating if data has been obtained via simulations have to be stored.
\end{multicols}

\noindent \begin{tabularx}{\columnwidth}{lllllX} 
\textbf{Parameter} & \textbf{Type} & \textbf{Dim} & \textbf{Unit/Format} & \textbf{Optional} & \textbf{Description} \\ \hline 
\inlvar{name} & \inltab{String} & 1 & & no & Experiment name \\ \hline
\inlvar{number} & \inltab{Int64} & 1 & & no & Experiment number within study\\ \hline
\inlvar{uuid} & \inltab{String} & 1 & f96dbc48-1ebd-44c7-b04d-1b45da054693 & no & Universally Unique Identifier (RFC 4122) of experiment \\ \hline 
\inlvar{description} & \inltab{String} & 1 & & no & Short description of the experiment \\ \hline
\inlvar{subject} & \inltab{String} & 1 & & no & Name of the subject that was imaged \\ \hline 
\inlvar{isSimulation} & \inltab{Int8} & 1 & & no & Flag indicating if the data in this file is simulated rather than measured \\ \hline
\end{tabularx}


\subsection{Tracer Parameters (group: \inlvar{/tracer/}, optional)}

\begin{multicols}{2}
\paragraph{Remarks:} The tracer parameter group contains information about the MPI tracers used during the experiment. For each of the $A$ tracers \inlvar{name}, \inlvar{batch}, \inlvar{vendor}, \inlvar{volume},  and molar \inlvar{concentration} of \inlvar{solute} per liter must be provided. Additionally, the time point of injection can be noted.

This version of the MDF can handle two basic scenarios. In the first one, static tracer phantoms are used. In this case, the phantom contains $A$ distinct tracers. For example, these might be particles of different core sizes, mobile or immobilized particles. In this case, \inlvar{injectionTime} is not used. In the second case, $A$ boli (e.g. pulsed boli) are administrated during the measurement, in which case the approximate administration volume, tracer type and time point of injection can be provided. Note that the injection clock recording the injection time should be synchronized with the clock, which provides the starting time of the measurement.

In case of a background measurement with no applied tracers in the scanner, the optional tracer group can be omitted.
\end{multicols}

\noindent \begin{tabularx}{\columnwidth}{lllllX} 
\textbf{Parameter} & \textbf{Type} & \textbf{Dim} & \textbf{Unit/Format} & \textbf{Optional} & \textbf{Description} \\ \hline 
\inlvar{name} & \inltab{String} & $A$ & & no & Name of tracer used in experiment \\ \hline
\inlvar{batch} & \inltab{String} & $A$ & & no & Batch of tracer \\ \hline
\inlvar{vendor} & \inltab{String} & $A$ & & no & Name of tracer supplier \\ \hline
\inlvar{volume} & \inltab{Float64} & $A$ & L & no & Total volume of applied tracer \\ \hline
\inlvar{concentration} & \inltab{Float64} & $A$ & mol(\inlvar{solute})/L & no & Molar concentration of \inlvar{solute} per litre \\ \hline
\inlvar{solute} & \inltab{String} & $A$ & & no & Solute, e.g. Fe \\ \hline
\inlvar{injectionTime} & \inltab{String} & $A$ & yyyy-mm-ddThh:mm:ss.ms & yes & UTC time at which tracer injection started \\ \hline
\end{tabularx}


\subsection{Scanner Parameters (group: \inlvar{/scanner/}, non-optional)}

\begin{multicols}{2}
\paragraph{Remarks:} The scanner parameter group contains information about the MPI scanner used, such as \inlvar{name}, \inlvar{manufacturer}, \inlvar{boreSize}, field \inlvar{topology}, \inlvar{facility} where the scanner is installed, and the \inlvar{operator}.
\end{multicols}

\noindent \begin{tabularx}{\columnwidth}{lllllX}
\noindent \textbf{Parameter} & \textbf{Type} & \textbf{Dim} & \textbf{Unit/Format} & \textbf{Optional} & \textbf{Description} \\ \hline 
\inlvar{name} & \inltab{String} & 1 & & no & Scanner name \\ \hline 
\inlvar{facility} & \inltab{String} & 1 & & no & Facility where the MPI scanner is installed \\ \hline 
\inlvar{operator} & \inltab{String} & 1 & & no & User who operates the MPI scanner \\ \hline 
\inlvar{manufacturer} & \inltab{String} & 1 & & no & Scanner manufacturer \\ \hline 
\inlvar{topology} & \inltab{String} & 1 & & no & Scanner topology (e.g. FFP, FFL, MPS)\\ \hline 
\inlvar{boreSize} & \inltab{Float64} & 1 & m & yes & Diameter of the bore \\ \hline 
\end{tabularx}

\newpage
\subsection{Acquisition Parameters (group: \inlvar{/acquisition/}, non-optional)}

\begin{multicols}{2}
\paragraph{Remarks:} The acquisition parameter group can describe different imaging protocols and trajectory settings. The corresponding data is organized into general information, a subgroup containing information on the $D$ excitation channels, and a subgroup containing information on the $C$ receive channels.

In MPI, a frame groups all data used to reconstruct a single MPI image/tomogram. In the simplest scenario, this data is acquired during one \inlvar{/drivefield/period}. The acquisition time increases by a factor of \inlvar{numAverages} if averaging is applied and by a factor of \inlvar{numPeriods} if data is recorded over several (averaged) cycles. In a multi-patch setting $J$ \inlvar{offsetField}s or mechanical movements shift the gradient field by \inlvar{offsetFieldShift}, where $J$ is the number of patches \inlvar{numPatches} of a multi-patch measurement. For each shift at least one full drive field cycle is used to acquire the measurement data. For instance, a Cartesian 2D trajectory with 100 lines could be realized by setting \mbox{\inlvar{numPatches} $ = 100$}.

\end{multicols}

\noindent \begin{tabularx}{\columnwidth}{lllllX}
\textbf{Parameter} & \textbf{Type} & \textbf{Dim} & \textbf{Unit/Format} & \textbf{Optional} & \textbf{Description} \\ \hline
\inlvar{startTime} & \inltab{String} & 1 & yyyy-mm-ddThh:mm:ss.ms & no & UTC start time of MPI measurement \\ \hline
\inlvar{framePeriod} & \inltab{Float64} & 1 & s & no & Time to complete a full frame (product of \inlvar{/drivefield/period}, \inlvar{numPeriods}, \inlvar{numAverages}, and \inlvar{numPatches}) \\ \hline
\inlvar{numPeriods} & \inltab{Int64} & 1 & 1 & no & Number of drive-field periods per patch \\ \hline
\inlvar{numAverages} & \inltab{Int64} & 1 & 1& no & Number of block averages per patch\\ \hline
\inlvar{numPatches} & \inltab{Int64} & 1 & 1 & no & Number of patches within a frame denoted by $J$ \\ \hline
\inlvar{numFrames} & \inltab{Int64} & 1 & 1& no & Number of available measurement frames $N$ \\ \hline
\inlvar{gradient} & \inltab{Float64} & $J \times 3$ & Tm$^{-1}\mu_0^{-1}$ & yes & Gradient strength of the selection field in $x$, $y$, and $z$ directions \\ \hline
\inlvar{offsetField} & \inltab{Float64} & $J \times 3$ & T$\mu_0^{-1}$ & yes & Offset field applied for each patch in the measurement sequence \\ \hline
\inlvar{offsetFieldShift} & \inltab{Float64} & $J \times 3$ & m & yes & Position of the field free point (relative to origin/center) \\ \hline
\end{tabularx}

\subsubsection{Drive Field (group: \inlvar{/acquisition/drivefield/}, non-optional)}

\begin{multicols}{2}
\paragraph{Remarks:} The drive field subgroup describes the excitation details of the imaging protocol. On the lowest level, each MPI scanner contains $D$ channels for excitation. Since most drive-field parameters may change from patch to patch, they have a leading dimension $J$.

These excitation signals are usually sinusoidal and can be described by $D$ amplitudes (drive field strengths), $D$ phases, a base frequency, and $D$ dividers. In a more general setting, the generated drive field of channel $d$ can be described by
$$
H_d(t) = \sum_{l=1}^{F} A_l \Lambda_l (2\pi f_l t + \varphi_l)
$$
where $F$ is the number of frequencies on the channel, $A_l$ is the drive-field strength, $\phi_l$ is the phase, $f_l$ is the frequency (\inlvar{baseFrequency}/\inlvar{divider}$_l$), and $\Lambda_l$ is the waveform. The waveform is specified by a dedicated parameter \inlvar{waveform}. It can be set to \textit{sine}, \textit{triangle}, or \textit{custom}. If set to \textit{custom}, one can specify a custom waveform using the parameter \inlvar{customWaveform}. The number of sampling points of the \inlvar{customWaveform} is denoted by $U$. The triangle is defined to be a $2\pi$ periodization of the
triangle function:
$$
 \Lambda_\text{tri}(t) = \left\vert t+\frac{\pi}{2}\right\vert - \frac{\pi}{2} \quad \text{with} \quad -\frac{3}{2}\pi\leq t \leq \frac{\pi}{2}.
$$
\end{multicols}

\noindent \begin{tabularx}{\columnwidth}{lllllX} 
\textbf{Parameter} & \textbf{Type} & \textbf{Dim} & \textbf{Unit/Format} & \textbf{Optional} & \textbf{Description} \\ \hline 
\inlvar{numChannels} & \inltab{Int64} & 1 & 1 & no & Number of drive field channels, denoted by $D$ \\ \hline
\inlvar{strength} & \inltab{Float64} & $J \times D \times F $ & T$\mu_0^{-1}$ & no & Applied drive field strength \\ \hline
\inlvar{phase} & \inltab{Float64} & $J \times D \times F$ & rad, $[-\pi,\pi)$ & no & Applied drive field phase $\varphi$\\ \hline
\inlvar{baseFrequency} & \inltab{Float64} & 1 & Hz & no & Base frequency to derive drive field frequencies \\ \hline
\inlvar{customWaveform} & \inltab{Float64} & $D \times F \times U$ & 1 & yes & Custom waveform table \\ \hline
\inlvar{divider} & \inltab{Int64} & $D \times F$ & 1 & no & Divider of the \inlvar{baseFrequency} to determine the drive field frequencies \\ \hline
\inlvar{waveform} & \inltab{String} & $D \times F$ & 1 & no & Waveform type: \textit{sine}, \textit{triangle} or \textit{custom} \\ \hline
\inlvar{period} & \inltab{Float64} & 1 & s & no & Trajectory period is determined by lcm(\inlvar{divider})/\inlvar{baseFrequency}\\ \hline
\end{tabularx}


\subsubsection{Receiver (group: \inlvar{/acquisition/receiver/}, non-optional)}

\begin{multicols}{2}
	\paragraph{Remarks:} The receiver subgroup describes the details on the MPI receiver. For a multi-patch sequence, it is assumed, that the signal acquisition only takes place during particle excitation. During each drive-field cycle, $C$ receive channels record some quantity related to the magnetization dynamic. In most cases these, will be a continuous voltage signals induced into the $C$ receive coils, which are proportional to the change of the particle magnetization. This analog signal will usually be converted by some sort of analog to digital converter (ADC) to a discrete series of integer numbers $r_{ci}$ for each channel $c$. To map the these values to the MPI measurement signal $u_{ci}^{ADC}$, one has to scale the numbers $r_{ci}$ and add an offset factor
\begin{equation*}
	u_{ci}^{ADC} = a_c r_{ci} +b_c.
\end{equation*}
Here, $a_c$ and $b_c$ are the scaling and offset factors corresponding to channel $c$, which can be stored in \inlvar{dataConversionFactor}. In case the conversion was already performed the \inlvar{dataConversionFactor} can be ommited.

The MPI measurement signal is acquired at $V$ equidistant time points. For inductive measurement systems the signal is usually not measured directly at the receive coils but amplified and filtered first, which may damp and distort the signal. Therefore, a transfer function can be stored in the parameter \inlvar{transferFunction}, which relates the Fourier domain voltage induced at the receive coil $\hat{u}_k^\text{coil}$ and the Fourier domain voltage $\hat{u}_k^\text{ADC}$ measured at the ADC by
\begin{equation*}
\hat{u}_k^\text{ADC} = a_k  \hat{u}_k^\text{coil}, \quad k=1,\dots,K.
\end{equation*}
Here, $a_k$ are the unitless parameters stored in \inlvar{transferFunction} for each receive channel individually. 

For MPS systems one can additionally store a parameter that maps the induced voltage to the magnetic moment of a magnetic nanoparticle located at the center of the scanner. More precisely, in each receive coil a projection of the magnetic moment onto the coil sensitivity is measured. The relation of this projection and the voltage at the receive coil in frequency space representation is given by
\begin{align*}
\hat{u}_k^\text{coil} = 2\pi \textrm{i} k \beta \hat{m}_k^\text{proj} , \quad k=1,\dots,K.
\end{align*}
where $\hat{m}_k^\text{proj}$ is the orthogonal projection of the full magnetic moment at the scanner center and $\beta$ is the channel dependent conversion factor that is stored in the parameter  \inlvar{inductionFactor}.
\end{multicols}

\noindent \begin{tabularx}{\columnwidth}{lllllX} 
\textbf{Parameter} & \textbf{Type} & \textbf{Dim} & \textbf{Unit/Format} & \textbf{Optional} & \textbf{Description} \\ \hline 
\inlvar{numChannels} & \inltab{Int64} & 1 & & no & Number of receive channels $C$ \\ \hline 
\inlvar{bandwidth} & \inltab{Float64} & $1$ & Hz & no & Bandwidth of the receiver unit \\ \hline
\inlvar{numSamplingPoints} & \inltab{Int64} & $1$ &  & no & Number of sampling points during one patch, denoted by $V$ \\ \hline
\inlvar{unit} & \inltab{String} & $1$ & & no & SI unit of the measured quantity, usually Voltage V \\ \hline 
\inlvar{dataConversionFactor} & \inltab{Float64} & $C \times 2$ & \inlvar{unit} & yes & Dimension less scaling factor and offset $(a_c, b_c)$ to convert raw data into a physical quantity with corresponding unit of measurement \inlvar{unit} \\ \hline 
\inlvar{transferFunction} & \inltab{Complex128} & $C \times K$ & 
& yes & Transfer function of the receive channels in Fourier domain. \inlvar{unit} is the field from the \inlvar{/measurement} group \\ \hline
\inlvar{inductionFactor} & \inltab{Float64} & $C$ & \inlvar{unit} A$^{-1}$m$^{-2}$ 
& yes & Induction factor mapping the projection of the magnetic moment to the voltage in the receive coil. \\ \hline
\end{tabularx}


\subsection{Measurement (group: \inlvar{/measurement/}, optional)}
\begin{multicols}{2}

\paragraph{Remarks:}
MPI data is usually acquired by a series of measurements and optional background measurements. Here, we refer to background measurements as MPI data captured, when any signal generating material, e.g. a phantom or a delta sample is removed from the scanner bore. Initially, all data is available in time domain, where the data of a single frame consists of the signal recorded for all patches in each receive channel, i.e. $J \times C \times V$ data points per set with the temporal index being the fastest to access.  If several measurements are acquired (indicated by \inlvar{numFrames}), the frame dimension is the slowest to access. Along this dimension, the frames are ordered with respect to the time at which they were acquired starting with the measurement acquired first and stopping with the measurement acquired last. We refer to this data as raw measurement data. In Fourier representation, each frame would be stored by $J \times C\times K$ complex data points and $K = V/2 +1$.

Often it is not convenient to store the raw data but to perform certain processing steps and store the processed data. These steps may lead to a reduction of the number of sampling points from $V$ to $W$ or a corresponding reduction of frequency components $K$ depending on the final representation in which the raw/processed data is stored. The most common processing steps are:
\begin{enumerate}
	\item Spectral leakage correction, which may be applied to ensure that each individual frame is periodic.
	\item Background correction, where the background signal in subtracted.
	\item Fourier transformation bringing the data from time into the Fourier representation an storing them in Fourier representation.
	\item Transfer function correction to obtain the magnetic moment or induced voltage that has been measured.
	\item Frequency selection to reduce the number of frequency components, e.g. bandwidth reduction or selection of high signal frequency components.
	\item Dimension permutation, which is usually applied to Fourier transformed data exchanging the storing order of the data for fast access to the frames.
	\item Frame permutation to reorder the frames within the data set.
\end{enumerate}
For each of the steps above there is a corresponding flag within this group indicating if the corresponding processing step has been carried out. 

During processing one might want to keep track which of the final $N$ frames belong to background measurements and which do not. Therefore, the binary mask \inlvar{isBackgroundFrame} can be used. If \inlvar{isBackgroundFrame} is not provided, it is assumed that no background measurements are present. If frequency selection has been performed, \inlvar{frequencySelection} stores the $K$ frequency components (subset) selected from the set of acquired frequency components. If performed, a frame permutation can be described by i.e. a bijective mapping $\sigma : \left\{ 1,2,\dots,N \right\} \rightarrow \left\{ 1,2,\dots,N \right\}$ of the set of frame indices to itself. If such a permutation is performed, $\sigma$ is stored in the one-line notation as $\sigma(1)$, $\sigma(2)$, $\dots$, $\sigma(N)$ in \inlvar{framePermutation}.

\end{multicols}

\noindent \begin{tabularx}{\columnwidth}{llp{3cm}lX} 
\textbf{Parameter} & \textbf{Type} & \textbf{Dim} &  \textbf{Optional} & \textbf{Description} \\ \hline 
\inlvar{data} & \inltab{Number} & $N \times J \times C \times K$ or $ J \times C \times K\times N$ or $N \times J \times C \times W$ or $ J \times C \times W \times N$ & no & Processed data \\ \hline
\inlvar{isBackgroundFrame} & \inltab{Int8} & $N$ & yes & Mask indicating for each of the $N$ frames if it is a background measurement (true) or not \\ \hline
\inlvar{isSpectralLeakageCorrected} & \inltab{Int8} & 1 & no & Flag, if spectral leakage correction has been applied \\ \hline
\inlvar{isBackgroundCorrected} & \inltab{Int8} & 1 & no & Flag, if the background has been subtracted \\ \hline
\inlvar{isFourierTransformed} & \inltab{Int8} & 1 & no & Flag, if the data is stored in frequency space \\ \hline
\inlvar{isTransferFunctionCorrected} & \inltab{Int8} & 1 & no & Flag, if the data has been corrected by the \inlvar{transferFunction}\\ \hline 
\inlvar{isFrequencySelection} & \inltab{Int8} & 1 & no & Flag, if only a subset of frequencies has been selected and stored, see \inlvar{frequencySelection}\\ \hline 
\inlvar{isPermuted} & \inltab{Int8} & 1 & no & Flag, if the frame dimension $N$ has been moved to the last dimension (second last for complex data) \\ \hline
\inlvar{isFramePermution} & \inltab{Int8} & 1 & no & Flag, if the order of frames has been changed, see \inlvar{framePermutation} \\ \hline 
\inlvar{frequencySelection} & \inltab{Int64} & K & \inlvar{isFrequencySelection} & Indices of selected frequency components \\ \hline
\inlvar{framePermutation} & \inltab{Int64} & $N$ & \inlvar{isFramePermutation} & Indices of original frame order\\ \hline
\end{tabularx} 

\newpage
\subsection{Calibration (group: \inlvar{/calibration/}, optional)}

\begin{multicols}{2}
\paragraph{Remarks:}
The calibration group describes a calibration measurement (system matrix), although it does not hold the data itself. Each of the measurements is taken with a calibration sample (delta sample) at a fixed position inside the FOV of the device. (If available, the background measurements are taken with the delta sample outside of the FOV of the scanner.) Usually, the calibration measurements are not stored as raw measurements but as processed data, where at least averaging, Fourier transformation, frame permutation and transposition of the data has been performed yielding $N$ processed calibration frames. $N$ includes the background measurements, whereas $O$ is the number of calibrated spatial positions scans. Which steps have been performed is documented in \inlvar{/measurement}.

If the measurements were taken on a regular grid of size $N_x \times N_y \times N_z$, the permutation is usually done such that measurements are ordered with respect to their $x$ position first, second with respect to their $y$ position, and last with respect to their $z$ position. Background measurements are collected at at the end in \inlvar{/measurement/data}, which in combination with reordering of the measurements allows a fast access to the system matrix. If a different storage order is used this can be documented using the optional parameter \inlvar{order}. For non-regular sampling points, there is the possibility to explicitly store all $O$ positions. In case of acquiring a hybrid system matrix, the spatial positions may be emulated by applying \inlvar{offsetFields} to the measurement chamber.
\end{multicols}

\noindent \begin{tabularx}{\columnwidth}{llp{3cm}llX} 
\textbf{Parameter} & \textbf{Type} & \textbf{Dim} & \textbf{Unit/Format} & \textbf{Optional} & \textbf{Description} \\ \hline 
\inlvar{method} & \inltab{String} & 1 & & no & Method used to obtain calibration data. Can for instance be robot, hybrid, or simulation \\ \hline
\inlvar{size} & \inltab{Int64} & $3$ &  & yes & Number of voxels in each dimension, inner product is $O$ \\ \hline
\inlvar{order} & \inltab{String} & 1 & & yes & Ordering of the dimensions, default is \textit{xyz} \\ \hline
\inlvar{positions} & \inltab{Float64} & $O \times 3$ & m & yes & Position of each of the grid points, stored as \mbox{($x$, $y$, $z$)} triples \\ \hline
\inlvar{offsetFields} & \inltab{Float64} & $O \times 3$ & T$\mu_0^{-1}$ & yes & Applied offset field strength to emulate a spatial position \mbox{($x$, $y$, $z$)}\\ \hline
\inlvar{deltaSampleSize} & \inltab{Float64} & 3 & m & yes & Size of the delta sample used for calibration scan \\ \hline
\inlvar{fieldOfView} & \inltab{Float64} & $3$ & m & yes & Field of view of the system matrix \\ \hline
\inlvar{fieldOfViewCenter} & \inltab{Float64} & $3$ & m & yes & Center of the system matrix (relative to origin/center) \\ \hline
\inlvar{snr} & \inltab{Float64} & $J \times C \times K$ &  & yes & Signal-to-noise estimate for recorded frequency components \\ \hline
\end{tabularx}

\newpage
\subsection{Reconstruction Results (group: \inlvar{/reconstruction/}, optional)}

\begin{multicols}{2}
Reconstruction results are stored using the parameter \inlvar{data} inside this group. \inlvar{data} contains a $Q\times P \times S$ array, where $Q$ denotes the number of reconstructed frames within the data set, $P$ denotes the number of voxels and $S$ the number of multispectral channels. If no multispectral reconstruction is performed, then one may set $S=1$. Depending on the reconstruction the grid of the reconstruction data can be different from the system matrix grid. Hence, grid parameters are mirrored in the \inlvar{/reconstruction} group.

For analysis of the MPI tomograms, it is often required to know which parts of the reconstructed tomogram have been covered by the trajectory of the field free region. In MPI, one refers to the non-covered region as overscan region. Therefore, the optional field \inlvar{isOverscanRegion} indicated for each voxel if it is part of the overscan region. If no voxel lies within the overscan region, \inlvar{isOverscanRegion} may be omitted.
\end{multicols}

\noindent \begin{tabularx}{\columnwidth}{lllllX} 
\textbf{Parameter} & \textbf{Type} & \textbf{Dim} & \textbf{Unit/Format} & \textbf{Optional} & \textbf{Description} \\ \hline 
\inlvar{data} & \inltab{Number} & $Q\times P \times S$ & & no & Reconstructed data \\ \hline
\inlvar{size} & \inltab{Int64} & $3$ &  & yes & Number of voxels in each dimension, inner product is $P$ \\ \hline
\inlvar{order} & \inltab{String} & 1 & & yes & Ordering of the dimensions, default is \textit{xyz} \\ \hline
\inlvar{positions} & \inltab{Float64} & $P \times 3$ & m & yes & Position of each of the grid points, stored as ($x$, $y$, $z$) tripels \\ \hline
\inlvar{fieldOfView} & \inltab{Float64} & $3$ & m & yes & Field of view of reconstructed data \\ \hline
\inlvar{fieldOfViewCenter} & \inltab{Float64} & $3$ & m & yes & Center of the reconstructed data (relative to scanner origin/center) \\ \hline 
\inlvar{isOverscanRegion} & \inltab{Int8} & $P$ &  & yes & Mask indicating for each of the $P$ voxels if it is part of the overscan region (true) or not \\ \hline
\end{tabularx}


\clearpage
\section{Changelog}

\begin{multicols}{2}
\subsection{v2.0.0}

\begin{itemize}
	\item Version 2 of the MDF is a major update breaking backwards compatibility with v1.x. The major update was necessary due to several shortcomings in the v1.x.
	\item The naming of parameters was made more consistent. Furthermore, some parameters move from one group into another.
	\item Defined a complex datatype using a HDF5 compound type. 
	\item In v1.x it was not possible to store background data. This functionality has been added in v2.
	\item We simplified the measurement group and made it much more expressive. In v1.x it was not entirely clear, which processing steps have been applied to the measurement data in the store dataset. The \inlvar{measurement} group now contains several flags that precisely describe the state of the stored data. Using this it is now possible to also store calibration data in the \inlvar{measurement} group. The calibration group in turn only stores metadata about a calibration experiment while the actual data is store in the measurement group.
	\item Updated Affiliations in the MDF specification.
	\item Improved the general descriptions of fields and groups.
	\item In v1.x the MDF allowed many fields to have varying dimensions depending on the context. As of version 2.0.0 only one field offers this freedom. This change should make implementations handling MDF files less complex.
	\item \inlvar{Number} has been introduced as generic types. The type restriction for several parameters has been weakened to these types.
	\item Added a table listing all variable names used in the descriptions of parameters.
	\item Added a section describing the possibility to add custom fields to MDF files.
	\item Added a description for optional and non-optional groups and conditional, optional, and non-optional datasets.
    \item Added a short section on the code examples on the Github repository.
	\item Support for triangle wave forms and fully arbitrary excitation waveforms has been added.
	\item Support for multiple excitation frequencies on a drive-field channel has been added.
	\item Added the dimension $A$ to all fields of the \inlvar{tracer} group to be able to describe settings where multiple tracers are used or tracers are administered multiple times.
	\item Added the possibility to store the tracer concentration also for non iron based tracer materials by adding the \inlvar{/tracer/solute} field and redefining the field \inlvar{tracer/concentration}.
 	\item Improved documentation for the storage of receive transfer functions. It is also possible now to store the measurement data as integer data and use a \inlvar{dataConversionFactor} to describe the mapping to a physical representation (e.g. Volt)
	\item Split the \inlvar{/study} group into the \inlvar{/experiment} group and the study group. This allows to provide more fine grained information on study and experiment.
	\item Added possibility to mark the overscan region.
	\item Added new section changelog to the MDF documentation to record the development of the MDF.
	\item Updated \inlvar{README.md} in the github repository.
	\item Updated code examples in the github repository.
	\item Added a section on the MDF reference implementation MPIFiles.jl. Since sanity checks will be covered by this package, the description on sanity checks has been removed.
\end{itemize}


\subsection{v1.0.5}

\begin{itemize}
	\item Added the possibility to store different channels of reconstructed data.
	\item Added support for receive channels with different characteristics (e.g. bandwidth).
	\item Made dataset \inlvar{/acquisition/receiver/frequencies} optional.
	\item Extended the description on the data types, which are used to store data.
	\item Added references for Julia and HDF5 to the specifications.
\end{itemize}


\subsection{v1.0.4}

\begin{itemize}
	\item Clarify that HDF5 datasets are used to store MPI parameters.
\end{itemize}


\subsection{v1.0.3}

\begin{itemize}
	\item Updated Affiliations in the MDF specification.
	\item Included data download into the Python and Matlab example code.
	\item Changes in the Python and Matlab example code to be better comparable to the Julia example code.
\end{itemize}


\subsection{v1.0.2}

\begin{itemize}
	\item Added reference to arXiv paper and bibtex file for reference.
\end{itemize}


\subsection{v1.0.1}

\begin{itemize}
	\item A sanity check within the Julia code shipped alongside the specifications.
	\item An update to the specification documenting the availability of a sanity check.
	\item Updated MDF files on https://www.tuhh.de/ibi/research/mpi-data-format.html.
	\item Updated documentation to the Julia, Matlab and Python reconstruction scripts.
	\item Improved Julia reconstruction script, automatically downloading the required MDF files.
\end{itemize}
\end{multicols}


\bibliographystyle{unsrt}
\bibliography{MDF}

\end{document}
